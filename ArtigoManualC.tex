\documentclass{article}
\usepackage[utf8]{inputenc}
\usepackage[lmargin=3cm, tmargin=3cm, rmargin=2cm, bmargin=2cm]{geometry} %layout
\usepackage[onehalfspacing]{setspace} %espaçamento das linhas
\usepackage[portuges, brazil]{babel}
\title{\Large{\textbf{Artigo Manual de Referência em C}}}
\author{Programa de Ensino e Tutorial Bacharelado em Ciência da Computação - UFSCar}

\begin{document}
\maketitle

% SEÇÃO INTRODUÇÃO
\section{Introdução}
\subsection{Motivação e objetivos}
\paragraph\normalcolor{O projeto Manual de Referência em C, elaborado e realizado pelo PET-BCC(Programa de Educação Tutorial de Bacharelado em Ciência da Computação) da UFSCar(Universidade Federal de São Carlos), campus São Carlos, tem como principal objetivo disponibilizar uma fonte de referência que seja confiável, intuitiva e escrita em português brasileiro sobre a linguagem de programação C. Dessa forma, o público alvo idealizado são os alunos dos cursos de BCC(Bacharelado em Ciência da Computação) e EnC(Engenharia de Computação) da federal, assim como intusiastas e estudantes da linguagem que ainda não possuem muita facilidade em consultar fontes em inglês ou que prefiram referências em português.}
\paragraph\normalfont{Para tanto, criou-se diversas páginas no site do PET-BCC contendo todos os cabeçalhos que são padrão da linguagem, com suas funções e demais ferramentas. Tais páginas, por sua vez, foram desenvolvidas com foco em terem um layout que fosse simples, de fácil navegação e que fornecesse de forma rápida a informação que o usuário veio a procura.}
\paragraph\normalfont{A concepção do Manual de Referência em C surgiu baseada na necessidade de fontes confiáveis e de fácil entendimento para refenciar a linguagem de programação C dada a sua importância para desenvolvedores de software e estudantes de programação e computação. Dessa forma, a disponibilização de um site com explicações curtas e intuitivas das diversas funções, com exemplos de uso, e demais ferramentas presentes nos cabeçalhos da linguagem C, alinhado à uma simples navegação contribui para suprir esta necessidade.}
% #TODO ver como se defini um termo que será repetido ao longo do artigo

% #TODO fonte 1 https://www.bell-labs.com/usr/dmr/www/chist.pdf
% #TODO fonte 2 https://insights.stackoverflow.com/survey/2020#learning--problem-solving
\paragraph\normalfont{}
\subsection{Concepção, uso e importância da linguagem C}
\paragraph\normalfont{No período entre os anos 1971 e 1973, o cientista da computação Dennis Ritchie criou a linguagem C a partir de uma linguagem de programação chamada B, idealizada por Ken Thompson, que por sua vez se inspirou na linguagem BCPL (\textit{Basic Combined Programming Language}), para a criação de B. Portanto, pode-se dizer que C é um fruto direto de B e de BCPL. Assim sendo, durante os anos de 1972 e 1977 Ritchie, Alan Snyder, Steven C. Johnson, Michael Lesk, and Thompson fizeram diversas contribuições à C, fazendo com que a coleção de bibliotecas, rotinas e cabeçalhos da linguagem crescesse muito. Por volta de 1978 Brian Kernighan e Ritchie escreveram o livro \textit{The C Programming Language}, considerado como a principal fonte de documentação da linguagem. Por fim, no começo de 1983, o comitê ANSI(\textit{American National Standard for Information Systems}) X3J11 realizou a primeira padronização da linguagem, gerando o que tem-se hoje como a biblioteca padrão de C.}
\paragraph\normalfont{C é uma linguagem de propósito geral, ela foi desenvolvida em paralelo com o sistema operacionai \textit{UNIX}, o qual é desenvolvido principalmente em C. Ademais, outras linguagens, como \textit{Python}, programas e ferramentas, como o banco de dados do \textit{Oracle} e \textit{Git}, são feitos sobre C, assim como diversos sistemas embarcados utilizam a linguagem. Em fevereiro de 2020 o \textit{Stack Overflow}, um dos principais sites mundiais para debates e fontes de pesquisa sobre tecnologia, realizou uma pesquisa com cerca de 65000 desenvolvedores de software, nela 33.1\% dos entrevistados afirmaram que trabalham e desejam continuar trabalhando com a linguagem C e que 4.3\% não utilizam a linguagem, mas possuem interesse em desenvolver aplicações com ela.}
\paragraph\normalfont{No contexto do DC(Departamento de Computação) da UFSCar São Carlos, departamento e campus respectivamente, sede do grupo PET-BCC, a linguagem C se faz importante logo no primeiro semestre dos cursos de Bacharelado em Ciência da Computação(BCC) e Engenharia de Computação(EnC). Isso é devido ao fato de a linguagem ser usado como ferramenta base do ensino da disciplina Construção de Algoritmos e programação, esta que é uma disciplina base de ambos os cursos citados. Além disso, C é usada em diversas outras disciplinas em BCC e EnC, logo oferecer uma fonte confiável para que os alunos(em especial os calouros, pois muitos entram na graduação com pouca ou nenhuma experiência em programação) possam consultar é de suma importância para a comunidade do DC.}

% SEÇÃO PLANEJAMENTO INICIAL
\section{Planejamento inicial do projeto}
\subsection{Design}
\subsection{Pesquisa}

% SEÇÃO IMPLEMENTAÇÃO FINAL
\section{Implementação final}
\subsection{Implementação}
\subsection{Design final das páginas}
\subsection{Bibliotecas implementadas}

% SEÇÃO CONSIDERAÇÕES FINAIS
\section{Considerações Finais}
\subsection{Objetivos atingidos}
\subsection{Análise do design final}
\subsection{Impacto na comunidade do Departamento de Computação}

\end{document}